%!TEX root = /Users/simo/Documents/PFC/memoria/memoria.tex
\section{Ruby on Rails} % (fold)
\label{sec:ruby_on_rails}

Tradicionalmente los desarrolladores web, ya sea en PHP, ASP, o cualquier otro lenguaje, han tenido sus propias metodologías de trabajo, sus librerías propias que han ido creando a lo largo de los años, malgastando así una cantidad de recursos enormes en implementar tareas que miles de personas ya han tenido que implementar con anterioridad. Incluso pudiendo obtener librerías con las que simplificar dichas tareas, la forma de enfocar el desarrollo de una aplicación no tenía la estructura común que aporta el trabajar sobre un framework concreto.

Trabajar sobre un framework de desarrollo aporta una serie de beneficios al desarrollador:

\begin{itemize}
  \item Procesos de desarrollos pautados. Al tener una estructura ya pactada, con las tareas típicas de desarrollo definidas, se ahorra tiempo y esfuerzo en planear los mismos. No solo eso, sino que al usar un framework popular es posible asumir que dichos procesos serán correctos, y pensados por desarrolladores que posiblemente tengan más experiencia en ello que uno mismo.
  \item Diferentes personas trabajan de la misma manera. El que diferentes personas estén utilizando los mismos procesos de desarrollo aporta varios puntos positivos.
  \begin{itemize}
    \item Es más fácil trabajar con otras personas, puesto todo el mundo sigue el mismo proceso, y no es necesario adaptar metodologías.
    \item Es más sencillo entender y trabajar con proyectos ya empezados, puesto que todos siguen la misma estructura.
    \item Al haber múltiples personas trabajando con las mismas herramientas, es más fácil encontrar errores y las soluciones que otras personas han compartido.
  \end{itemize}
\end{itemize}

El primer punto ayuda a desarrollar código de mayor cálidad y con menos errores, tanto por el hecho de que, desde un principio se sabe que se están siguiendo metodologías correctas, y porque al tener la mayoría de problemas menores resueltos, es posible dedicarle una mayor atención a las partes que realmente importan.

Rails es el framework más conocido para Ruby, pero existen múltiples alternativas:

\begin{description}
  \item[Merb\footnote{\url{http://merbivore.com/}}] Escrito en \textbf{Ruby}, y que se fusionará con Rails próximamente.
  \item[Django\footnote{\url{http://www.djangoproject.com/}}] Escrito en \textbf{Python}.
  \item[CakePHP\footnote{\url{http://cakephp.org/}}] Escrito en \textbf{PHP}.
  \item[Struts\footnote{\url{http://struts.apache.org/}}] Escrito en \textbf{Java}.
\end{description}

La sensación en estos momentos es que se está tendiendo a utilizar este tipo de frameworks a la hora de desarrollar proyectos nuevos, tanto personal como comercialmente. Struts,

% section ruby (end)