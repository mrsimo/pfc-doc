%!TEX root = /Users/simo/Documents/PFC/Chapter4/chapter4.tex
\section{Esta aplicación en particular} % (fold)
\label{sec:esta_aplicacion_en_particular}

La aplicación está online de forma pública \footnote{\url{http://github.com/albertllop/pfc}} desde el hosting de repositorios GIT (un sistema de control de versiones) github \footnote{\url{http://github.com}}. Es posible descargar el código mediante el enlace de descarga incluido en la página de la aplicación, o descargando el código directamente mediante git:

\begin{verbatim}
  $ git clone git://github.com/albertllop/pfc.git
\end{verbatim}

Se recomienda la segunda metodología para facilitar una actualización futura más sencilla del código. La mayoría de pasos a seguir a partir de aquí son comunes a cualquier otra aplicación en Ruby on Rails.

\subsection{Configurar la base de datos} % (fold)
\label{sub:configurar_la_base_de_datos}

Es necesario especificar los datos para la conexión a una base de datos. El archivo en cuestión se encuentra en \texttt{/config/database.yml}. Se ha incluido un archivo de ejemplo para facilitar la tarea, en \texttt{/config/database.yml.template}

\begin{verbatim}
  # database.yml.template
  development:
    adapter: mysql
    encoding: utf8
    database: drawme_development
    username: root
    password: root
    host: localhost
    socket: /temp/mysql.sock

  production:
    adapter: mysql
    encoding: utf8
    database: drawme_production
    username: root
    password: root
    host: localhost
    socket: /temp/mysql.sock
\end{verbatim}

Como se observa, es posible configurar diferentes conexiones y bases de datos tanto para el entorno de desarrollo como para el de producción. Ruby on Rails soporta MySql, PostgreSQL y SQLite de forma nativa, aunque existen múltiples Gems y tutoriales para poder conectar con la mayoría de SGBD existentes \footnote{\url{http://wiki.rubyonrails.com/rails/pages/HowtosDatabase}}, como Oracle, Microsoft SQL Server o IBM DB2.

Una vez configurada la base de datos todos los procesos a seguir se pueden ejecutar desde la línea de comandos. Se debe crear la base de datos en si, y generar la estructura de tablas:

\begin{verbatim}
  rake db:create RAILS_ENV=production
  rake db:migrate RAILS_ENV=production
\end{verbatim}

El parámetro \texttt{RAILS\_ENV=production} es necesario cuando se quiere ejecutar cualquier tarea de \texttt{rake} suponiendo el entorno de producción, puesto que el entorno por defecto es el de desarrollo.

% subsection configurar_la_base_de_datos (end)

\subsection{Arrancando los mongrels} % (fold)
\label{sub:arrancando_los_mongrels}

Puesto que la combinación de Balanceador Proxy de Apache con varios procesos Mongrel es una de las formas más habituales de servir aplicaciones Ruby on Rails, se han incluido unas tareas \texttt{Rake} para ayudar con el proceso. Se debe editar el archivo de configuración \texttt{/config/mongrel\_cluster.yml} y cambiar dos valores, \texttt{port} y \texttt{servers}:

\begin{verbatim}
  # mongrel_cluster.yml
  --- 
  port: 3000
  servers: 2
  log_file: log/mongrel.log
  environment: production
  pid_file: tmp/pids/mongrel.pid  
\end{verbatim}

El parámetro \texttt{port} sirve para indicar el puerto en que iniciar los procesos, y el parámetro \texttt{servers} para indicar cuantos procesos iniciar. En este caso, se iniciarían dos instancias mongrel en los puertos \texttt{3000} y \texttt{3001}.

Las tareas en si son las siguientes:

\begin{verbatim}
  rake servers:start
  rake servers:stop
  rake servers:restart
\end{verbatim}

Las cuales inician, paran, o reinician los procesos automáticamente.

% subsection arrancando_los_mongrels (end)

\subsection{Configurando SSL} % (fold)
\label{sub:configurando_ssl}

Una aplicación Ruby on Rails no difiere de cualquier otra en cuanto a su soporte para SSL. Dado que, en los ejemplos contemplados, se presupone una configuración con Apache recibiendo las peticiones (tanto con FastCGI, \texttt{mod\_rails} o Mongrels), es posible implementar el soporte para conexiones seguras directamente en la configuración del site de Apache. El procedimiento más sencillo es duplicando la configuración incluyendo el puerto de SSL y los parámetros adecuados.

\begin{verbatim}
  <VirtualHost *:443>
    SSLEngine on
    SSLCertificateFile /etc/ssl/certs/cert.pem

    DocumentRoot /path/to/application/public

    RewriteEngine On

    <Proxy balancer://app>
      BalancerMember http://127.0.0.1:3000
      BalancerMember http://127.0.0.1:3001
    </Proxy>
    
    # Redirect all non-static requests
    RewriteCond %{DOCUMENT_ROOT}/%{REQUEST_FILENAME} !-f
    RewriteRule ^/(.*)$ balancer://app%{REQUEST_URI} [P,QSA,L]

    ProxyPass / balancer://app/
    ProxyPassReverse / balancer://app/
    ProxyPreserveHost on

  </VirtualHost>
\end{verbatim}

De este modo se tendría soporte en toda la aplicación para SSL. Puesto que se ha establecido que el soporte SSL sería opcional, no se requieren más pasos. En caso de querer añadir SSL como un requisito para usar alguna funcionalidad de la web, o que esté disponible solo en algunas acciones, se podría realizar mediante el plugin \texttt{ssl\_requirement} \footnote{\url{http://dev.rubyonrails.org/svn/rails/plugins/ssl\_requirement/}}, soportado oficialmente por Ruby on Rails, pero no incluido en el núcleo de funcionalidades de Rails.

\begin{verbatim}
  # Instalar el plugin
  $ script/plugin install ssl_requirement
  
  # Requerir SSL para las acciones login y panel del controlador user_controller.rb
  ssl_required :login, :panel
  
  # Ofrecer la posibilidad de utilizar SSL para las acciones login y panel del
  # controlador user_controller.rb
  ssl_allowed :login, :panel
\end{verbatim}

% subsection configurando_ssl (end)










