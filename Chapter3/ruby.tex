%!TEX root = /Users/simo/Documents/PFC/Chapter3/chapter3.tex
Se dispone en estos momentos de un motor completo en Javascript que permitiría todo el proceso necesario para los objetivos de este proyecto. Éste, sin embargo, necesita de un fondo que le aporte todo lo que html estático y javascript no es capaz de hacer, como por ejemplo, tratar con la base de datos. Por las razones comentadas en el primer capítulo, se ha considerado que Ruby on Rails sería la solución para implementar este fondo, y en este capítulo se comenta el proceso seguido, resaltando en cada caso las diferencias de utilizar Ruby on Rails frente a soluciones más clásicas como PHP o ASP.

\section{Consideraciones previas} % (fold)
\label{sec:ruby_on_rails_consideraciones_previas}
Antes de nada, para entender como funciona Ruby on Rails, hay que aclarar ciertos conceptos. Primero, el lenguaje en que se está programando es Ruby. Éste es un lenguaje interpretado, que al igual que lenguajes de scripting típicos como Perl, PHP o Python, se compilan dinámicamente a la hora de la ejecución. De hecho, existen implementaciones de interpretadores de Ruby en varios lenguajes, siendo la implementada en C la \emph{oficial}, pero existiendo otras tan dispares como jRuby, una implementación en Java que permite utilizar cualquier librería de Java.

Ruby es un lenguaje de alto nivel, con una sintaxis que hace que sea muy fácil de entender para gente que no la conoce. Por ejemplo:

\begin{verbatim}
100.times do 
  print "No hablaré en clase".upcase
end  
\end{verbatim}

La popularidad de Ruby se ha incrementado con el éxito de Ruby on Rails, y ha hecho que se puedan encontrar librerías para prácticamente cualquier cosa. La sencillez del código lo hacen muy adecuado para el mundo del desarrollo web, donde se prioriza la agilidad de desarrollo, antes que una gran eficiencia. Existen múltiples técnicas de cacheo que hacen que el procesado necesario por el lenguaje de scripting sea mínimo, relegando todo el trabajo al servidor web (Apache, por ejemplo), y a unas consultas a la base de datos bien eficientes. Por tanto, lenguajes de más bajo nivel que podrían aportar una mayor eficiencia del código no se consideran adecuados por requerir de un proceso de desarrollo más largo y costoso, para unas ganancias relativamente mínimas. El uso de este tipo de lenguajes se relega a partes muy pequeñas y precisas, normalmente en los culos de botella donde puedan ser útiles. Se considera más beneficioso poder desarrollar más, de forma más sencilla para así evitar bugs indeseables, y utilizar dichas técnicas de cacheo y de gestión de base de datos para lograr la eficiencia.

Rails es un Framework escrito en ruby que permite un desarrollo web ágil y sencillo. Se basa en hacer fácil el trabajo del programador, y en su famoso \emph{Convention over configuration} (convención antes que configuración). Debido a que la mayoría del tiempo, el desarrollo de webs es un proceso repetitivo, es posible establecer una serie de patrones que asumir, y solo modificar en los casos especiales en que \emph{lo normal} no es suficiente.

Está basado en una arquitectura MVC (Modelo Vista Controlador), permitiendo aplicar el patrón en 3 capas estudiado las diversas asignaturas de Ingeniería del Software, así como la mayoría de buenas prácticas, hasta ahora mayoritariamente difíciles de aplicar en el mundo del desarrollo web.

Rails, gracias a la sencillez de Ruby, promueve la práctica del desarrollo ágil de software (Agile software development), una metodología de desarrollo que se basa en aligerar el proceso de desarrollo, alejarse de metodologias \emph{burocráticas}, centrándose en conseguir software de alta calidad de forma muy rápida. Estas características son muy apreciadas en el mundo del desarrollo web, puesto que como ya se ha comentado, la eficiencia suele ser secundaria, y los procesos, al ser repetitivos en su mayoría, hacen de otras metodologías demasiado pesadas y lentas.

A lo largo de este capítulo se irán remarcando los puntos por los cuales Ruby es tan adecuado al desarrollo web, porqué Rails permite un desarrollo más ágil, y porqué una metodología basada en la escasez de documentación y planificación es posible, y de hecho, beneficiosa, en el contexto de este proyecto (y en el de otros similares de desarrollo de webs). 

% section ruby_on_rails_consideraciones_previas (end)
