%!TEX root = /Users/simo/Documents/PFC/Chapter3/chapter3.tex
Se dispone en estos momentos de un motor completo en Javascript que permitiría todo el proceso necesario para los objetivos de este proyecto. Éste, sin embargo, necesita de un fondo que le aporte todo lo que html estático y javascript no es capaz de hacer, como por ejemplo, tratar con la base de datos. Por las razones comentadas en el primer capítulo, se ha considerado que Ruby on Rails sería la solución para implementar este fondo, y en este capítulo se comenta el proceso seguido, resaltando en cada caso las diferencias de utilizar Ruby on Rails frente a soluciones más clásicas como PHP o ASP.

\section{Consideraciones previas} % (fold)
\label{sec:ruby_on_rails_consideraciones_previas}
Antes de nada, para entender como funciona Ruby on Rails, hay que aclarar ciertos conceptos. Primero, el lenguaje en que se está programando es Ruby. Éste es un lenguaje interpretado, que al igual que lenguajes de scripting típicos como Perl, PHP o Python, no es necesario compilar. De hecho, existen implementaciones de interpretadores de Ruby en varios lenguajes, siendo la implementada en C la \emph{oficial}, pero existiendo otras tan dispares como jRuby, una implementación en Java que permite utilizar cualquier librería de Java.

Desde un principio su creador pretendió hacer de ruby un lenguaje muy \emph{natural} a la hora de ser leído, y lleva al extremo el concepto de lenguaje de alto nivel. Entre sus características se encuentra el considerar absolutamente un objeto (incluyendo tipos básicos), permitiendo generar líneas tan autoexplicativas como las siguientes:
\begin{verbatim}
100.times do 
  print "No hablaré en clase".upcase
end  
\end{verbatim}

Incluso con la naturalidad con la que se se escribe y se lee Ruby, no hace de él un lenguaje simple, y es lo suficientemente robusto y funcional como para permitir escribir cualquier tipo de aplicación. Sin embargo, allá donde ha triunfado más, es en el mundo del desarrollo web, donde cada vez más, se busca la agilidad de desarrollo más que una gran eficiencia del código. Las características de las webs hacen que sea posible lograr una gran eficiencia una vez una web está funcionando, mediante métodos como el cacheo, que hacen que la mejora de eficiencia que se pueda lograr con otros lenguaje de más bajo nivel, sean minúsculas.

Rails es un Framework escrito en ruby que permite un desarrollo web mucho más ágil y sencillo. Se base en hacer fácil el trabajo del programador, y en su famoso \emph{Convention over configuration} (convención antes que configuración). Debido a que la mayoría del tiempo, el desarrollo de webs es un proceso repetitivo, es posible establecer una serie de patrones que asumir, y solo modificar en los casos especiales en que \emph{lo normal} no es suficiente.

Está basado en una arquitectura MVC (Modelo Vista Controlador), permitiendo aplicar el patrón en 3 capas estudiado las diversas asignaturas de Ingeniería del Software, así como la mayoría de buenas prácticas, hasta ahora prácticamente imposibles de aplicar en el mundo web.

Rails (en parte gracias a la sencillez de Ruby), promueve la práctica del desarrollo ágil de software (Agile software development), una metodología de desarrollo que se basa en aligerar el proceso de desarrollo, alejarse de metodologias \emph{burocráticas}, centrándose en conseguir software de alta calidad de forma muy rápida. Estas características son muy apreciadas en el mundo del desarrollo web, puesto que como ya se ha comentado, la eficiencia suele ser secundaria, y los procesos, al ser repetitivos en su mayoría, hacen de otras metodologías demasiado pesadas y lentas.

A lo largo de este capítulo se irán remarcando los puntos por los cuales Ruby es tan adecuado al desarrollo web, porqué Rails permite un desarrollo mucho más fácil, y porqué una metodología basada en la escasez de documentación y planificación es posible, y de hecho, beneficiosa, en el contexto de este proyecto (y en el de otros similares de desarrollo de webs). 

% section ruby_on_rails_consideraciones_previas (end)

\section{Primer ciclo} % (fold)
\label{sec:primer_ciclo}
La metodología de desarrollo ágil defiende un desarrollo iterativo, por ciclos perqueños que generen de forma rápida ciertas funcionalidades. Cada ciclo amplica funcionalidades hasta que al final estén todas incluidas. Cada ciclo debe ser completo, con sus fases de análisis de requisitos, diseño, codificación y revisión. La fase de documentación tiene como resultado este capítulo.

En este primer ciclo, se pretende obtener todas las funcionalidades necesarias para poder crear y editar documentos.

\subsection{Análisis de requisitos} % (fold)
\label{sub:análisis_de_requisitos}

Gracias a que se ha tratado la implementación del Javascript de forma previa a este ciclo, se conocen ya los requisitos básicos necesarios para poder utilizar el motor de dibujo. De firna detallada, las funcionalidades requeridas para este ciclo son las siguientes:

\begin{itemize}
  \item Poder crear, editar y borrar documentos, dándoles un título, una descripción y asignándole una cantidad de páginas fija, mediante interacción normal por HTML.
  \item Tener soporte para documentos, páginas y los elementos contenidas en ellas, de forma que el motor de comunicación en Javascript pueda comunicarse con la aplicación para editar las pizarras.
\end{itemize}

% subsection análisis_de_requisitos (end)


% section primer_ciclo (end)